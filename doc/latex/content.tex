\newpage

Svet sa neustále mení. Menia sa aj ľudia, ich návyky, potreby a~spôsob života. Mení sa aj pohľad na svet, ktorý sa vďaka neustále sa rozvíjajúcim informačným a~komunikačným technológiám čoraz viac zmenšuje. To, čo bolo kedysi neuveriteľné a~nepredstaviteľné, sa teraz deje prakticky v~každej domácnosti. Ľudia sa dokážu behom desatiny sekundy spojiť s~kýmkoľvek na opačnej strane zemegule a~to všetko vďaka vylepšujúcim sa komunikačným technológiám. Kým kedysi na prenesenie informácie bolo treba fyzicky vyslať posla so správou na určené miesto, dnes sa to deje lusknutím prstu. A~to ešte stále nie je koniec.

Keď sa pozrieme do minulosti, môžeme si všimnúť, že tento celý vývoj narastá exponenciálne a~to najmä v~posledných rokoch našej existencie. Keď sa nad tým zamyslíme, je až desivé pomyslieť, čo tu môže byť o~dvadsať rokov. Vedci prednedávnom vyvinuli smrteľne nebezpečnú superchrípku\footnote{http://veda.sme.sk/c/6202878/vedci-vytvorili-zakernu-superchripku.html}, ktorá by za okamih dokázala vyhladiť celé ľudstvo. Vďaka dnešnej rozvinutej možnosti komunikácie a~okamžitému prístupu k~internetu by sa pomaly ktokoľvek mohol dozvedieť recept na tento extrémne nebezpečný vírus. Keď k~tomu pričítame ľudskú nenažranosť po moci, peniazoch a~iných prostriedkoch, je tu možnosť, že by sme sa za chvíľu mohli sami zničiť. To je už ale extrémny pohľad na vec.

Keď sa na to pozrieme z~tej lepšej stránky, neustále sa rozvíjajúce informačné a~komunikačné technológie nám napomáhajú ku kvalitnejšiemu životu a~zlepšujú náš celkový komfort. V~istých situáciách nám dokonca môžu zachrániť náš život.

Ako príklad by sa dala uviesť udalosť, ktorá sa stala v~roku 2007, kedy osoba menom Tanya Rider išla sama autom do práce, keď náhle zišla z~cesty a~zrútila sa priamo do rokliny. Osem dní bola uväznená vo vraku svojho auta. Nesmierne dehydrovaná a~so zraneniami nohy a~ramena takmer skonala na zlyhanie obličiek. Našťastie ju záchranári predsa len našli. Tanya strávila mesiace zotavujúc sa v~zdravotníckych zariadeniach. Nakoniec bola schopná vrátiť sa domov ešte do Vianoc. Tanya bola nájdená, pretože telefónne spoločnosti si uchovávajú záznamy o~polohách mobilných telefónov. Keď má niekto so sebou svoj mobilný telefón, tak ten pravidelne vysiela digitálny signál, niekoľko bitov oznamujúcich ``Hej, tu som!''. Telefón posiela tieto správy tak dlho, pokiaľ je zapnutý. Vďaka tomu mala telefónna spoločnosť stále u~seba zaznamenaný údaj poslednej pozície Tanyinho telefónu, dokonca aj potom, čo sa jej telefón rozbil. Takto ju teda našla polícia.\cite{B2B}

Ja osobne som dostal svoj prvý mobilný telefón v~prvom ročníku na strednej škole. Bol to pre mňa obrovský pokrok v~komunikácii, nakoľko predtým som sa vedel na diaľku dohovoriť iba cez fixne zabudovanú pevnú linku. Každý postupne dostával vlastný mobil, čím sa od vtedy prakticky jednoznačne identifikoval pre diaľkovú komunikáciu.

S~postupom času sa taká istá situácia diala aj s~osobnými počítačmi. Dnes už prakticky každý obyvateľ tejto planéty má svoj osobný počítač. S tým opäť vznikajú pokročilejšie spôsoby a~metódy komunikácie v~porovnaní s~mobilnými telefónmi. Okrem zvuku je už možné prenášať aj obraz, čo sa už náramne blíži k~osobnému kontaktu.

Ako je vidieť, informačné a~komunikačné technológie naozaj hýbu svetom a~neustále ho menia a~transformujú do iných dimenzií. Tento vývoj nemôžme len tak ignorovať, týka sa to nás všetkých a~musíme sa s~tým naučiť žiť. Ostáva už len čakať, čo so sebou prinesie budúcnosť\dots